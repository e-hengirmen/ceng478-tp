For parallel matrix multiplication of 2 N$^2$ matrices A and B to create a N$^2$ matrix C there are 3 basic methods that comes to mind.
\begin{enumerate}
	\item Every process will calculate rows of the C matrix
	\item Every process will calculate columns of the C matrix
	\item Every process will calculate a block of the C matrix
\end{enumerate}
Even though in the end there will be exactly the same number of calculations the first 2 methods has an inefficient side opposed to the 3rd method. The problem with the first 2 methods comes from the way the matrix multiplication works. If we are using matrix row method then every process will only need that corresponding row from the A matrix but since the elements in a rows are in every column then every process will need all of the B matrix to process their rows. \\
\\
so every process(assume there are P number of processes) would need to get N/P rows from matrix A and all of the B matrix which means to send the matrices we ould need to make 1 Broadcast operation and 1 scatter operation from process 0. Broadcast would take O($log(p)(t_s+$N$^2t_m$)) time (since it is the whole matrix the message size is N$^2$). On the other hand scatter operation would be done on rows and would take O(\(log(p)t_s+(p-1)\frac{N^2}{p}t_m\)) time ($\frac{N^2}{p}$ is the message size every process recevives) so in the end just sending the matrices would take O(N$^2$log(p)) operations(because we need to broadcast the whole matrix)\\
\\
On the other hand for the block approach we will only need to send rows from A matrix and columns from B matrix normally and since we chose cannon's algorihm we will only send \textbf{blocks} of $\frac{N}{\sqrt{p}}\times \frac{N}{\sqrt{p}}$ instead. Which would need 2 scatter operations of complexity O(\(log(p)t_s+(p-1)\frac{N^2}{p}t_m\))).
so the complexity of initialization would be O(\(N^2\)) for this case.\\
\\
\subsection{Cannon's Algorithm}
The problem with Block based matrix multiplication is the fact that when we try to calculate an entry of the result matrix's value(for example c$_{ik}$) we tend to start from $a_{i0}$ and $b_{0k}$. Which means that every process would need to start with left-most A group and upper-most B group. (For example in the matrix below the Group 1 2 and 3 of C matrixes would need to start with Group1 of A.)
\[ 
\begin{bmatrix}
Group1 & Group2 & Group3\\
Group4 & Group5 & Group6\\
Group7 & Group8 & Group9\\
\end{bmatrix}
\]
Cannon's algorithm uses a way to bypass this problem. By their method instead of sending every group of block of matrix to their correspongind process we would instead rotate every block of A by its block row and every block of B by its block column according to their block row number and block column number. In this way at any given time, each process would be using a different block in every iteration(while keeping the computation accurate). For example assuming the above matrix is A it would become
\[ 
\begin{bmatrix}
Group1 & Group2 & Group3\\
Group5 & Group6 & Group4\\
Group9 & Group7 & Group8\\
\end{bmatrix}
\]
and if it was the B matrix it would become
\[ 
\begin{bmatrix}
Group1 & Group5 & Group9\\
Group4 & Group8 & Group3\\
Group7 & Group2 & Group6\\
\end{bmatrix}
\]
\textbf{Note:} We are not actually swapping the blocks then sending them. Instead we are sending corresponding A and B blocks to neccessary processes.\\
\\
After Initial block sending from process 0 is done every program will need to calculate their given blocks. But to calculate C block of for example group 9 the program would need to make $\sqrt{p}$ block matrix multiplications (for Group9 it would calculate AGroup8*BGroup6+AGroup7*BGroup3+AGroup9*BGroup9). To do this every process will send its received blocks(A and B blocks) to the next process and receive 2 blocks from the previous one for $\sqrt{p}-1$ times($\sqrt{p}$ blocks in every row and column but since the first one was received from process 0 we only need to make $\sqrt{p}-1$ iterations.)\\
\\
Sending these messages would take $t_s+\frac{N^2}{P}t_m$ for every iteration so in total sending and receiving messages to next and from previous partners would take $2(\sqrt{p}-1)(t_s+\frac{N^2}{P}t_m)$\\
\\
And for every $\frac{N^2}{P}$ entry we would need to make N multiplications and N-1 additions so in total $\frac{2N^3-N^2}{P}$ operations would be done paralelly for every matrix and the time complexity of computations would become O($\frac{N^3}{P}$)\\
\\
The last part is receiving these matrices. To do this the program would only need to make a gather operations to receive blocks from every process to one of the specified processes. And that would take O(\(log(p)t_s+(p-1)\frac{N^2}{p}t_m\))) time\\
\\
So in the end Cannon's Algorithm(Parallel) would make 3 scatter/gather operations(2 scatters for distributing A and B matrixes and 1 gather to gather calculated C matrix) in \(3log(p)t_s+3(p-1)\frac{N^2}{p}t_m\) time. Parallely sends blocks to each partner processes which takes $2(\sqrt{p}-1)(t_s+\frac{N^2}{P}t_m)$ time and calculates the C block of the matrix in $\frac{2N^3-N^2}{P}$ time which comes to \(\frac{2N^3-N^2}{P}+t_s(3log(p)+2\sqrt{p}-1)+t_m\frac{N^2}{P}(3p+2\sqrt{p}-4)\) or O(\(max(\frac{N^3}{P},N^2)\))\\
\\
Ant the sequential algorithm's time only depends on computations which comes up to N$^3$-N$^2$ or O($N^3$)\\
\\
\textbf{Note:} Because of the scatter and gather operations time complexity will never exceed O(N$^2$). And increasing the number of processes is not always efficient.The problem is that just sending the matrix itself would take O(N$^2$) opearions so thinking that this is the best case.