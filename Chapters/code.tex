There are 2 c files in the src directory \textbf{serial.c} and \textbf{solution.c}. \textbf{solution.c} has the parallel algorithm while \textbf{serial.c} has the sequential one.\\
\\
To use the codes use \textbf{script.sh} inside the src directory.It works in this order:
\begin{itemize}
	\item It will compile \textbf{solution.c}(with 4 processors u can change it by going inside script.sh) and \textbf{serial.c}
	\item Then it will start parallel algorithm
		\subitem Parallel algorithm will first create A and B matrices
		\subitem Then copy them inside \textbf{input} file(for sequential algorithm's use)
		\subitem At this point it starts Parallel timer
		\subitem After calculating the solutions it will first write the time it took to calculate C matrix in seconds.
		\subitem After that it will record the C matrix inside the \textbf{output} file
	\item Then it will run sequential algorithm with the inputs from parallel algorithm
		\subitem Sequential algorithm will start the timer after getting inputs and stop the timer and print it after calculation is done
		\subitem after printing the timer sequential algorithm will save the C matrix inside \textbf{output2} file.
	\item then the script will call diff command with output and output2
\end{itemize}
\textbf{Important Note:} Before using them make sure serial. \textbf{solution.c} and \textbf{serial.c} both has the same defined value for N or they will produce different results because of their different matrix sizes.