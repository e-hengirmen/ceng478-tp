Problems from 1-4 are problems we need to solve to parallelize the Linear Programming models. But here, we will focus on simultaneously solving multiple Linear Programming problems that are created from an Integer Programming problem instead.\\
\\
To solve an integer programming problem, the program would need to solve and create multiple linear programming problems. After every solution, the program either finds a valid solution or branches into 2 new problems. Furthermore, since branching is almost always the case for the initial steps, the problem count will increase exponentially for some time. Because of this, the complexity of the sequential algorithm is $2^n$.\\
\\
However, we can improve the time complexity by reducing the problem count from \textbf{1+2+4+8...} to \textbf{1+1+1+1...} by solving them simultaneously, as we said in the introduction section. However, there is an important problem. In the case of constant branching, the program would need to solve 1048576 problems at the 20th step (This will rarely happen because we can remove problems that can not get a better solution than our current best solution.). \\
\\
Since we can not have infinite processes, the program should check if the process count is optimal, at every step. We will solve these problems simultaneously only if the processor count is enough to solve these problems without losing efficiency on Linear problems. Otherwise the program should focus on solving the optimal amount of problems. The reason for that is the fact that when we get a valid better solution, we can remove problems from L whose z$_u^L$ is lower then newly calculated z$L$. In this case focusing on only a group of Linear Problems would be optimal.